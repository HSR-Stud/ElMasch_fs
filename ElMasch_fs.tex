\documentclass{article}

%definitions for Formelsammlung

\usepackage[left=1.5cm,right=1.5cm,top=2.5cm,bottom=2cm,landscape]{geometry} 
\usepackage{multicol}
\usepackage[ngerman]{babel}
\usepackage{tabularx}
\usepackage{mathpazo}
\usepackage{mathtools}
\usepackage{amsmath}  
\usepackage{setspace} 
\usepackage{commath}
\usepackage[utf8]{inputenc}
%\usepackage[ansinew]{inputenc}  
\usepackage[T1]{fontenc}
\usepackage{lmodern} 
\usepackage{hyperref}
\usepackage{bigints}
\usepackage{array}
\usepackage[table]{xcolor}
\usepackage{layouts}
\usepackage{siunitx}
\usepackage{wrapfig}
\usepackage{multirow,bigstrut}
\usepackage{trfsigns}
\usepackage{amssymb} 
\usepackage{fancyhdr}
\usepackage{datetime}
\usepackage{pgfplots}
\usepgfplotslibrary{fillbetween}
\usepackage{listings} 
\usepackage{mathrsfs}
\usepackage{booktabs}
\usepackage[f]{esvect}


\DeclareMathOperator\arctanh{arctanh}
\DeclareMathOperator\arsinh{arsinh} 
\DeclareMathOperator\arcosh{arcosh}
\DeclareMathOperator\artanh{artanh}
\DeclareMathOperator\arcoth{arcoth} 
\DeclareMathOperator\sinc{sinc} 
\DeclareMathOperator\sgn{sgn} 
\DeclareMathOperator\LPF{LPF} 
\DeclareMathOperator\Q{Q} 
\DeclareMathOperator\erf{erf} 


%colorCodes
\definecolor{listinggray}{gray}{0.9}
\definecolor{lbcolor}{rgb}{0.95,0.95,0.95}
\definecolor{lightGray}{gray}{0.1}

\definecolor{cOrange}{HTML}{996633}
\definecolor{clOrange}{HTML}{DBB48D}
\definecolor{cBlue}{HTML}{336699}
\definecolor{clBlue}{HTML}{A0BCD8}
\definecolor{cGreen}{HTML}{339966}
\definecolor{clGreen}{HTML}{94D4B4}
\definecolor{cRed}{HTML}{993333}
\definecolor{clRed}{HTML}{D0B0B0}
\definecolor{cGray}{gray}{0.4}
\definecolor{clGray}{gray}{0.96}



\setlength{\parindent}{0pt}
%\DeclareMathOperator\arctanh{arccot}
\newcolumntype{L}[1]{>{\raggedright\let\newline\\\arraybackslash\hspace{0pt}}m{#1}}
\newcolumntype{C}[1]{>{\centering\let\newline\\\arraybackslash\hspace{0pt}}m{#1}}
\newcolumntype{R}[1]{>{\raggedleft\let\newline\\\arraybackslash\hspace{0pt}}m{#1}}
\newcolumntype{Y}{>{\centering\arraybackslash}X}
\newcolumntype{Z}{>{\raggedleft\arraybackslash}X}
\newcommand{\fmm}{\displaystyle} 
\newcommand{\cn}[1]{\underline{#1}} 
\newcommand{\hlaplace}{\quad\laplace\quad}
\newcommand{\hLaplace}{\quad\Laplace\quad}
\newcommand{\infint}{\int_{-\infty}^{+\infty}}
\newcommand{\infiint}{\iint_{-\infty}^{+\infty}}
\newcommand{\limint}{\lim_{T\rightarrow \infty} \frac{1}{T} \int_{-T/2}^{T/2}}
\newcommand{\bedeq}{\mathrel{\stackrel{\makebox[0pt]{\mbox{\normalfont\tiny WSS}}}{=\joinrel=}}}

%\renewcommand{\vec}[1]{\vv{#1}}

\renewenvironment{description}{\color{cGray}}{}
\newenvironment{definition}{\color{cGray}}{}
\newcommand{\cdef}[1]{\begin{definition}#1\end{definition}}
\newcommand{\cunit}[2]{\left[\frac{#1}{#2}\right]}

\newcommand{\clCell}[2]{\cellcolor{cl#1}{\color{c#1}#2}}

\newcommand{\circled}[1]{\tikz[baseline=(char.base), scale=0.8, transform shape]{
                          \draw (0,0) circle (0.2) node (char) {#1};}}


\newcommand{\vLaplace}[1][]{\mbox{\setlength{\unitlength}{0.1em}%
        \begin{picture}(10,20)%
          \put(3,2){\circle{4}}%
          \put(3,4){\line(0,1){12}}%
          \put(3,18){\circle*{4}}%
          \put(10,7){#1}
        \end{picture}%
       }%
 }%

\newcommand{\vlaplace}[1][]{\mbox{\setlength{\unitlength}{0.1em}%
        \begin{picture}(10,20)%
          \put(3,2){\circle*{4}}%
          \put(3,4){\line(0,1){12}}%
          \put(3,18){\circle{4}}%
          \put(10,7){#1}
        \end{picture}%
       }%
 }%                    
 
 
 
\renewcommand{\arraystretch}{1.5}

\newenvironment{mtabular}[1] {
  \renewcommand{\arraystretch}{2}
  
  \begin{tabular}{#1}
}  
{
  \end{tabular}
  
  \renewcommand{\arraystretch}{1.5}
}

\newenvironment{dtabular} {
  \begin{tabular}{>{\begin{definition}}l<{\end{definition}} >{\begin{definition}}l<{\end{definition}}}
}  
{
  \end{tabular}
}


%configure tikz
%system description
\usetikzlibrary{shapes,arrows, patterns, fadings}
\tikzstyle{block} = [draw, rectangle, minimum height=3em, minimum width=4em]
\tikzstyle{input} = [coordinate]
\tikzstyle{output} = [coordinate]
\tikzstyle{pinstyle} = [pin edge={to-,thin,black}]
\tikzstyle{sum} = [draw, circle, node distance=1em, minimum height=1.5em]


\usepackage[american]{circuitikz}
\ctikzset{resistor = european}


%lstlisting

\lstset{
  backgroundcolor=\color{lbcolor},
  tabsize=2,    
% rulecolor=,
  language=[GNU]C++,
  basicstyle=\scriptsize,
  upquote=true,
  aboveskip={1.5\baselineskip},
  columns=fixed,
  showstringspaces=false,
  extendedchars=false,
  breaklines=true,
  prebreak = \raisebox{0ex}[0ex][0ex]{\ensuremath{\hookleftarrow}},
  frame=single,
  numbers=none,
  showtabs=false,
  showspaces=false,
  showstringspaces=false,
  identifierstyle=\ttfamily,
  keywordstyle=\color{cBlue}
  commentstyle=\color{cGreen},
  stringstyle=\color{cRed},
  numberstyle=\color{black},
% \lstdefinestyle{C++}{language=C++,style=numbers}’.
}
\lstset{
  backgroundcolor=\color{lbcolor},
  tabsize=2,
  language=C++,
  captionpos=b,
  tabsize=3,
  frame=lines,
  numbers=none,
  numberstyle=\tiny,
  numbersep=5pt,
  breaklines=true,
  showstringspaces=false,
  basicstyle=\ttfamily,
  identifierstyle=\color{cOrange},
  keywordstyle=\color{cBlue},
  commentstyle=\color{cGreen},
  stringstyle=\color{cRed}
}

\lstdefinelanguage{makefile}{
  morekeywords={cc,CFLAGS,LFLAGS,OBJ,EXE},
  morecomment=[l]{\#}
}

\lstdefinestyle{makefile}{
  language=makefile,
  basicstyle=\ttfamily,
  keywordstyle=\color{cBlue},
  commentstyle=\color{cGreen},
  frame=lines,
  numbers=none,
  backgroundcolor=\color{lbcolor}
}

%header & footer
\pagestyle{fancy}
\lhead{Tibor Schneider}
\rhead{Seite \thepage}
\cfoot{\today} 

\renewcommand{\headrulewidth}{0.4pt}
\renewcommand{\footrulewidth}{0.4pt}
%Title of Document
\chead{Elektrische Maschinen} 

\begin{document}
\begin{twocolumn} 

\section{Felder}
\subsection{Elektrisches Feld}

Folgende Formeln gelten für 2 Dimensionen. Dazu müssen die Ladungsträger zylinderförmig 
sein. 

\begin{tabular}{ll}
  \begin{tabular}{ll}
    \cdef{$\varepsilon \cunit{As}{Vm}$} & \cdef{dielektrische permittivität} \\
    \cdef{$Q, q \cunit{C}{m}$} & \cdef{Linienladungsdichte} \\
    \cdef{$\vec{r}_0$} & \cdef{Einheitsvektor}\\
    \cdef{$\vec{F}_e \cunit{N}{m}$} & \cdef{Elektrische Kraft} \\
    \cdef{$\vec{E} \cunit{V}{m}$} & \cdef{Elektrisches Feld} \\
  \end{tabular} &
  \begin{tabular}{l}
    $\varepsilon = \varepsilon_0 \cdot \varepsilon_r, \quad \varepsilon_0 = 8.8542 \cdot
    10^{-12} $
    \\
    $\fmm \vec{F}_e = \frac{Q \cdot q}{2 \pi \varepsilon r} \cdot \vec{r}_0$ \\
    $\fmm \vec{E} = \frac{\vec{F}_e}{q} = \frac{Q}{2 \pi \varepsilon r} \cdot \vec{r}_0$ \\
    $\fmm U_{AB} = \int_A^B \vec{E} \cdot \vec{dl} = \varphi_A - \varphi_B$
  \end{tabular}\\
\end{tabular}

\subsection{Magnetisches Feld}

Für die Richtungen der Vektoren eines Kreuzproduktes kann folgende Regel angewendet 
werden: $\vec{a}$: Daumen, $\vec{b}$: Zeigefinger, $\vec{a} \times \vec{b}:$ Mittelfinger

\begin{tabular}{ll}
  \begin{tabular}{ll}
    \cdef{$\mu \cunit{N}{A^2}$} & \cdef{magnetische permeabilität} \\
    \cdef{$I, i [A]$} & \cdef{Ströme der Leiter} \\
    \cdef{$\vec{F}_m \cunit{N}{m}$} & \cdef{magnetische Kraft} \\
    \cdef{$\vec{H} \cunit{A}{m}$} & \cdef{magnetisches Feld} \\
    \cdef{$\vec{B}$ [T]} & \cdef{magnetische Flussdichte} \\
    \cdef{$\Phi [Wb]$} & \cdef{magnetischer Fluss} \\
    \cdef{$\Psi [Wb]$} & \cdef{verketteter mag. Fluss} \\
    \cdef{$\Theta [A]$} & \cdef{magnetische Durchflutung} \\
    \cdef{$V_m [A]$} & \cdef{Magnetische Spannung} \\
    \cdef{$R_m \cunit{A}{Wb}$} & \cdef{Magnetischer Wiederstand} \\
    \cdef{$\gamma$} & \cdef{geschlossener Weg um Leiter} \\
    \cdef{$l, r$} & \cdef{Länge / Radius einer Spule} \\
    \cdef{$W_m [J]$} & \cdef{Magnetische Energie}
  \end{tabular} &
  \begin{tabular}{l} 
    $\mu = \mu_0 \cdot \mu_r \quad \mu_0 = 4 \pi \cdot 10^{-7}$ \\
    $\fmm \vec{F}_m = \frac{\mu}{2\pi} \cdot \frac{I i}{r}\cdot \vec{r}_0 = \mu \cdot i \cdot \vec{l}_0 \times \vec{H}$ \\
    $\fmm \vec{H} = \frac{I}{2\pi} \cdot \frac{\vec{L}_0 \times \vec{r}_0}{r}$ \\
    $\fmm \vec{B} = \mu \cdot  \vec{H}$ \\
    $\fmm \Phi = \iint_A \vec{B} \cdot \vec{dA}, \quad \Psi = N \cdot \Phi$ \\
    $\fmm U_{ind} = - \frac{d\Psi}{dt} = -N \cdot \frac{d\Phi}{dt}$ \\
    $\fmm \Theta = \sum_{k=1}^n I_k = \oint_\gamma \vec{H} \cdot \vec{dl} = N \cdot I$  \\ 
    $\fmm V_m = \int_A^B \vec{H} \cdot \vec{dl}$ \\
    $\fmm R_m = \frac{V_m}{\Phi} = \frac{l}{\mu \cdot A} \text{(wenn Homogen)}$ \\
    $\fmm H = \frac{N \cdot I}{l}, \quad  \text{wenn } l \gg r$ \\
    $\fmm W_m = \frac{1}{2} \iiint_V B \cdot H \cdot dV =  \frac{1}{2} \cdot H \cdot B \cdot V$
  \end{tabular}
\end{tabular}

\subsection{einfacher Magnetkreis}

\begin{tabular}{ll}
  \begin{tabular}{l}
    \begin{tikzpicture}
      \filldraw[fill=clGray] (0,0) -- node[above, xshift=-1.7cm, yshift=-0.05cm]{Eisenjoch}(5,0) -- (5,1) -- (0,1) -- cycle;
      \filldraw[fill=clGray] (0,1.5) -- (1,1.5) -- (1,3) -- (4,3) -- (4,1.5) -- (5,1.5) -- (5,4) -- node[below, xshift=-1.7cm, yshift=0.05cm]{Eisenkern}(0,4) -- cycle;
      \draw[preaction={fill, clRed}, pattern=north west lines, pattern color=cRed] (1.5, 2.95) rectangle (3.5,2.6);
      \draw[preaction={fill, clRed}, pattern=north west lines, pattern color=cRed] (1.5, 4.05) rectangle (3.5,4.4);
      \draw[>-<, thin](-0.1,1.5) -- node[right, xshift=-2pt]{\small $\delta/2$} (-0.1,1);
      \draw[>-<, thin](5.1,1.5) -- node[left, xshift=2pt]{\small $\delta$/2} (5.1,1);
      \draw[>=latex, ->, draw=cRed, thick] (3,3.5) -- node[above]{$\color{cRed}\Phi$} (2,3.5);
      \draw[dashed, cRed] (2,3.5) -- (0.5,3.5) -- node[right]{$\color{cRed}\gamma$}(0.5,1.5);
      \draw[dashed, cGreen] (0.5,1.5) -- (0.5,1);
      \draw[dashed, cRed] (0.5,1) -- (0.5,0.5) -- (4.5,0.5) -- (4.5,1);
      \draw[dashed, cGreen] (4.5,1) -- (4.5,1.5);
      \draw[dashed, cRed] (4.5,1.5) -- (4.5,3.5) -- (3, 3.5);
      \draw[>=latex, ->, draw=cBlue, thick] (2.5,0.7) -- node[right]{$\color{cBlue}\vec{F}_R$} (2.5,2);
    \end{tikzpicture}
  \end{tabular} &
  \begin{tabular}{l}
    Im Luftspalt: $\mathbf{B}$ \textbf{Konstant!} \\ 
    $\fmm \oint_{\gamma} \vec{H} \vec{dl} = H_{Fe} \cdot l_{Fe} + 2 \delta H_\delta = N I$ \\
    \cdef{Länge aller Luftspalte $\delta$} \\
    $\fmm H_\delta = \frac{N \cdot I}{\delta}, \textrm{ wenn } H_\delta \gg H_{Fe}$ \\
    $\fmm F_R = \frac{\partial W_m}{\partial \delta} = \mu_0 \cdot \frac{N^2 I^2 A_{Fe}}{4\delta^2}$ \\
    $\fmm W_m = \frac{1}{2} H_\delta B_\delta \cdot 2 A_{Fe} \delta = \frac{\mu_0 A_{Fe} I^2 N^2}{4 \delta}$ \\
    \textbf{in der Sättigung:} \\
    $\fmm H_\delta = \frac{N \cdot I}{\frac{\mu_0}{\mu_{Fe}} l_{Fe} + 2 \delta}$
  \end{tabular} \\
\end{tabular}

\subsection{Dauermagnet}
\begin{tabular}{ll}
  \begin{tabular}{l}
    \begin{tikzpicture}
      \begin{axis}[
        width=5cm, 
        height=5cm, 
        axis lines=middle, 
        xlabel={$H \cunit{A}{m}$},
        ylabel={$B [T]$}, 
        x label style={at={(axis description cs:1.37,0.03)}},
        y label style={at={(axis description cs:0.73,1.15)}},
        xmin=-1, xmax=0.2, 
        ymin = -0.2, ymax = 1, 
        xtick={-0.8}, 
        xticklabels={$H_C$}, 
        ytick={0.8}, 
        yticklabels={$B_R$}]
        
        \draw[cRed, thick](axis cs:-0.83, -0.03) -- (axis cs:0.03, 0.83);
        
      \end{axis}
    \end{tikzpicture}
  \end{tabular}
  \begin{tabular}{l}
    \cdef{Koerzitivfeldstärke $H_C$} \\
    \cdef{Remanenz $B_R$} \\
    $B_m = \mu_m \cdot H_m + B_R, \quad \mu_m = \mu_0$ \\
    \textbf{Magnetkreis mit Dauermagnet} \\
    \cdef{Magnet: $l_m$, $\mu_m$}, im Eisenjoch \\
    $\fmm H_\delta = \frac{N \cdot I + \frac{B_r}{\mu_m} l_m}{\frac{\mu_0}{\mu_{Fe}}l_{Fe} + \frac{\mu_0}{\mu_m}l_m + 2\delta}$
  \end{tabular}
\end{tabular}

\section{Gleichstrommaschine}

\begin{tabular}{ll}

  \begin{tabular}{l}
    \begin{tikzpicture} [scale=0.8, every node/.style={scale=0.8}]
      \draw [fill=clGray] (0,0) circle (4);
      \draw (0,3.4) node{Ständer}; 
      \draw [fill=white] (0,0) circle (3);
      \draw [fill=clGray] (0,0) circle (1.3);
      
      \draw [fill=clGray, draw=none] (-0.5,3) -- (0,3.3) -- (0.5,3) -- (0.5,2) -- (1.3, 1.2) -- (1.3,1) -- (1.2,0.9) arc (36.87:143.13:1.5) -- (-1.2,0.9) -- (-1.3,1) -- (-1.3,1.2) -- (-0.5,2) -- (-0.5,3);
      \draw (0.5,2.958) -- (0.5,2) -- (1.3, 1.2) -- (1.3,1) -- (1.2,0.9) arc (36.87:143.13:1.5) -- (-1.2,0.9) -- (-1.3,1) -- (-1.3,1.2) -- (-0.5,2) -- (-0.5,2.958);
      
      \draw [fill=clGray, draw=none] (-0.5,-3) -- (0,-3.3) -- (0.5,-3) -- (0.5,-2) -- (1.3, -1.2) -- (1.3,-1) -- (1.2,-0.9) arc (-36.87:-143.13:1.5) -- (-1.2,-0.9) -- (-1.3,-1) -- (-1.3,-1.2) -- (-0.5,-2) -- (-0.5,-3);
      \draw (0.5,-2.958) -- (0.5,-2) -- (1.3, -1.2) -- (1.3,-1) -- (1.2,-0.9) arc (-36.87:-143.13:1.5) -- (-1.2,-0.9) -- (-1.3,-1) -- (-1.3,-1.2) -- (-0.5,-2) -- (-0.5,-2.958);
      
      \draw[preaction={fill, clRed}, pattern=north west lines, pattern color=cRed] (0.55, 2.05) rectangle (1.15,2.65);
      \draw[preaction={fill, clRed}, pattern=north west lines, pattern color=cRed] (-0.55, 2.05) rectangle (-1.15,2.65);
      \draw[preaction={fill, clRed}, pattern=north west lines, pattern color=cRed] (0.55, -2.05) rectangle (1.15,-2.65);
      \draw[preaction={fill, clRed}, pattern=north west lines, pattern color=cRed] (-0.55, -2.05) rectangle (-1.15,-2.65);
    
      \foreach \phi in {10,30,...,170}
        {\draw [fill=clBlue] (\phi:1.15) circle (0.115);
         \draw [fill=cBlue, draw=cBlue] (\phi:1.15) circle (0.03);} 
      \foreach \phi in {190,210,...,350}
        {\draw [fill=clBlue] (\phi:1.15) circle (0.115) node{$\color{cBlue}\times$};} 
        
      \draw [fill=cGray, draw=cGray] (1.3,0.2) rectangle (1.5,-0.2);
      \draw [fill=cGray, draw=cGray] (-1.3,0.2) rectangle (-1.5,-0.2);
      \draw [draw=cGray] (1.5,0) -- (2.25,0) node[above]{$-$};
      \draw [draw=cGray] (-1.5,0) -- (-2.25,0) node[above]{$+$}; 
      \draw [fill=cGray, draw=cGray] (2.25,0) circle (0.06);
      \draw [fill=cGray, draw=cGray] (-2.25,0) circle (0.06);
      
      %references
      \draw (0,2.6) circle (0.2) node{1};
      \draw (0.5,-1.66) circle (0.2) node{2};
      \draw (2.475, 2.475) circle (0.2) node{3};
      \draw (-1.3, 2.05) circle (0.2) node{4};
      \draw (1.35, 0.6) circle (0.2) node{5};
      \draw (-1.6,-0.35) circle (0.2) node{6};
      \draw (-0.6,-0.5) circle (0.2) node{7};
      
      %field
      \draw[->, >=stealth', draw=cRed](170:3.5) arc (170:190:3.5);
      \draw[->, >=stealth', draw=cRed](10:3.5) arc (10:-10:3.5);
      \draw[->, >=stealth', draw=cRed] (0,-2.8) -- node[left, xshift=2]{\color{cRed}$\vec{B}_e$} (0,-2);
      \draw[->, >=stealth', draw=cRed] (0,0) -- node[left, xshift=2]{\color{cRed}$\vec{B}_e$} (0,0.8);
      \draw[->, >=stealth', draw=cBlue] (0,0) -- node[below, yshift=1]{\color{cBlue}$\vec{B}_a$} (0.4,0);
      \draw[->, >=stealth', draw=cGreen] (0,0) -- node[right]{\color{cGreen}$\vec{B}_g$} (0.4,0.8);
    
    \end{tikzpicture}
    
  \end{tabular} &
  
  \begin{tabular}{l}
    \begin{circuitikz} [scale=0.7, transform shape]
      \draw (0,0) to [short, *-] (1.5,0) to [voltage source, l={\large $E$}] (1.5,1.5);
      \draw (0,4.5) to [short, *-,  i={\large $I_a$}] (1.5,4.5) to [resistor, l_={\large $R_a$}] (1.5,3) to [inductor, l_={\large$L_a$}] (1.5,1.5);
      \draw[-latex,shorten >=2mm, shorten <=2mm,in=110, out=250] (0,4.5) to node[right] {$\cn{U_a}$} (0,0);
      \draw (4,4.5) to [short, *-, i_={\large $I_e$}] (2.5,4.5) to [resistor, l={\large $R_e$}] (2.5,3) to [inductor, l={\large $L_e$}] (2.5,1.5) to [short, -*] (4,1.5);
      \draw[-latex,shorten >=2mm, shorten <=2mm,in=70, out=290] (4,4.5) to node[left] {\large $\cn{U_e}$} (4,1.5);
    \end{circuitikz} \\
    \circled{1}: Polkern, 
    \circled{2}: Polschuh \\
    \circled{3}: Ständerjoch \\
    \circled{4}: Erregerwicklung (EW) \\
    \circled{5}: Ankerwicklung (AW) \\
    \circled{6}: Bürste, 
    \circled{7}: Anker
  \end{tabular}
  
\end{tabular}

\begin{tabular}{ll}
  \begin{tabular}{ll}
    \cdef{$M [Nm]$} & \cdef{Drehmoment} \\
    \cdef{$P [W]$} & \cdef{Leistung} \\
    \cdef{$X_a$} & \cdef{Anker-grösse} \\
    \cdef{$X_e$} & \cdef{Erreger-Grösse} \\
    \cdef{$\vec{B}_a$} & \cdef{Ankerrückwirkung} \\
    \cdef{$n \cunit{1}{min}$} & \cdef{Drehzahl}
  \end{tabular}
  
  \begin{tabular}{l}
    $U_e = R_e \cdot I_e + L_e \cdot \frac{dI_e}{dt}$ \\
    $U_a = R_a \cdot I_a + L_a \cdot \frac{dI_a}{dt} + E$ \\
    $E = \omega \cdot \Psi \qquad \omega = \frac{2\pi}{60} \cdot n \qquad \Psi = L_e \cdot I_e$
    \\
    $P_{el} = \underbrace{ R_e \cdot I_e^2 }_{\text{Erregerverluste}} + \underbrace{R_a \cdot I_a^2}_{\text{Ankerverluste}} + \omega \cdot \Psi \cdot I_a$ \\
    $P_{mech} = \omega \cdot M, \qquad M = \Psi \cdot I_a$ \\
    $\fmm M = \frac{U_a \cdot \Psi - \omega \cdot \Psi^2}{R_a} \qquad I_a = \frac{U_a-\omega \Psi}{R_a}$ \\
  \end{tabular}
\end{tabular}
 
\subsection{Kompensation der Ankerrückwirkung}
\begin{itemize}
  \item Die \textbf{Kompensationswicklung (KW)} wird im Polschuh des Stators eingesetzt. 
  Somit wirkt ein Feld $\vec{B}_{kw}$ gegen die Ankerrückwirkung $\vec{B}_{a}$. Die 
  Nuten werden durch den Polschuh geführt.
  \item Die \textbf{Compoundwicklung (KP)} gleicht die durch die Nuten der KW verursachte
  Hauptfeldschwächung aus. Diese wird in Serie zu der EW montiert.
  \item Durch die KP wird das Feld $\vec{B}_e$ verstärkt. Somit stimmt das Gleichgewicht 
  von $\vec{B_g}$ nicht mehr. Deshalb wird die \textbf{Wendepolwicklung (WW)} eingesetzt.
  Sie wird im Ständerjoch montiert, so dass $\vec{B}_{ww}$ gegen $\vec{B}_{a}$
  zeigt.
\end{itemize}
 
\subsection{Beschaltung}
\subsubsection{Nebenschluss}
Hier werden Erreger- und Ankerwicklung parallel an die gleiche Spannungsquelle geschaltet. 
Somit gilt: $U_e = U_a = U$. \cdef{$M_A$: Anlaufmoment, $n_0$: Leerlaufdrehzahl, $I_{aA}$: Anlaufstrom, $R_v$: Anlaufwiederstand}

$$M = I_a \cdot \Psi = \frac{U \cdot \Psi - \frac{2\pi}{60} n \cdot \Psi^2}{R_a} \qquad I_a = \frac{U - \frac{2\pi}{60}n \cdot \Psi}{R_a} $$
$$n = 0 \Rightarrow M_A = \frac{U \cdot \Psi}{R_a} \qquad I_{aA} = \frac{U}{R_a + R_v} \qquad M=0 \Rightarrow n_0 = \frac{U}{\frac{2\pi}{60} \Psi}$$

\begin{tabular}{ll}
  \begin{tabular}{l}
    \begin{tikzpicture}
      \begin{axis} [
        width=6cm, 
        height=4cm, 
        axis lines=middle, 
        ylabel={$M [Nm]$},
        xlabel={$n \cunit{1}{min}$}, 
        xmin=0, xmax=1.2, 
        ymin = 0, ymax = 1.4, 
        xtick={0.45,0.9}, 
        xticklabels={$n_b$,$n_0$}, 
        ytick={0.45,0.9}, 
        yticklabels={$M_b$, $M_a$}
      ]
      
      \draw [cRed, thick] (axis cs:0.9,0) -- (axis cs:0,0.9);
      \draw [dashed] (axis cs:0.45,0) -- (axis cs:0.45,0.45);
      \draw [dashed] (axis cs:0.45,0.45) -- (axis cs:0,0.45);
      \draw [fill=black] (axis cs:0.45,0.45) circle (0.05cm);
        
      \end{axis}
    \end{tikzpicture}
  \end{tabular} &
   
  \begin{tabular}{l}
    $\fmm \frac{M}{M_A} = 1 - \frac{n}{n_0}$ \\ 
    \cdef{$M_B$: Betriebsmoment}\\
    \cdef{$n_b$: Betriebsdrehzahl} \\
  \end{tabular}
\end{tabular}

\subsubsection{Reihenschluss}
Hier werden Erreger- und Ankerwicklungen in Serie an die gemeinsame Spannungsquelle 
geschaltet. Nun Gilt: $I_e = I_a = I$. \cdef{$M_A$: Anlaufmoment, $n_b$: 
Bezugsdrehzahl}

$$U = U_a + U_e = (R_a + R_e) I + \frac{2\pi}{60}n \cdot \Psi, \qquad \Psi = L_e \cdot I$$
$$M = I \cdot \Psi = L_e \left(\frac{U}{R_a + R_e + \frac{2\pi}{60}n L_e}\right)^2$$
$$n=0 \Rightarrow M_A = \frac{L_e \cdot U^2}{(R_a + R_e)^2} \qquad n_b = \frac{R_a + R_e}{\frac{2\pi}{60} L_e}$$

\begin{tabular}{ll}
  \begin{tabular}{l}
    \begin{tikzpicture}
      \begin{axis} [
        width=6cm, 
        height=4cm, 
        axis lines=middle, 
        ylabel={$M [Nm]$},
        xlabel={$n \cunit{1}{min}$}, 
        xmin=0, xmax=1.2, 
        ymin = 0, ymax = 1.2, 
        xtick={0.45}, 
        xticklabels={$n_b$}, 
        ytick={0.47,1}, 
        yticklabels={$M_b$, $M_a$}
      ]
      
      \addplot [draw=cRed, thick, domain=0:4*pi, samples=200]{1/((1+(x))^2)};
      
      \draw [dashed] (axis cs:0.45,0) -- (axis cs:0.45,0.47);
      \draw [dashed] (axis cs:0.45,0.47) -- (axis cs:0,0.47);
      \draw [fill=black] (axis cs:0.45,0.47) circle (0.05cm);
        
      \end{axis}
    \end{tikzpicture}
  \end{tabular} &
  
  \begin{tabular}{l}
    $\fmm \frac{M}{M_A} = \frac{1}{\left(1+\frac{n}{n_b}\right)^2}$ \\ 
    \cdef{$M_B$: Betriebsmoment}\\
    \cdef{$n_b$: Betriebsdrehzahl} \\
  \end{tabular}
\end{tabular}

\section{Drehfelderzeugung}

\begin{tabular}{ll}
  \begin{tabular}{l}
    \begin{tikzpicture} [scale=0.8, every node/.style={scale=0.8}]
      \draw[fill=clGray] (0,0) circle (4);
      \draw[fill=white] (0,0) circle (2.6);
      \draw[fill=clGray] (0,0) circle (2.5);
      
      \foreach \phi/\col/\x in {90/Blue/U_{1+}, 75/Blue/U_{2+}, 60/Blue/U_{3+}, 45/Blue/U_{4+},
                                -30/Green/V_{1+}, -45/Green/V_{2+}, -60/Green/V_{3+}, -75/Green/V_{4+},
                                -150/Red/W_{1+}, -165/Red/W_{2+}, 180/Red/W_{3+}, 165/Red/W_{4+}}
      {
%         \draw[fill=cl\col] (\phi:2.8) circle (0.16) node{\color{c\col}$\times$};
        \draw[fill=white, draw=none] (\phi-3.5:2.55) -- (\phi-3.5:2.9) -- (\phi+3.5:2.9) -- (\phi+3.5:2.55) arc (\phi+3.5:\phi-3.5:2.55);
        \draw(\phi-3.5:2.6) -- (\phi-3.5:2.9) -- (\phi+3.5:2.9) -- (\phi+3.5:2.6);
        \draw[fill=cl\col](\phi-2.5:2.6) -- (\phi-2.5:2.85) -- (\phi+2.5:2.85) -- (\phi+2.5:2.6) -- (\phi-2.5:2.6);
        \draw[thick, c\col](\phi-2.5:2.6) -- (\phi+2.5:2.85);
        \draw[thick, c\col](\phi-2.5:2.85) -- (\phi+2.5:2.6);
        \draw(\phi:3.3) node {\color{c\col}$\x$};
      }
      
      \foreach \phi/\col/\x in {30/Red/W_{1-}, 15/Red/W_{2-}, 0/Red/W_{3-}, -15/Red/W_{4-},
                                -90/Blue/U_{1-}, -105/Blue/U_{2-}, -120/Blue/U_{3-}, -135/Blue/U_{4-},
                                150/Green/V_{1-}, 135/Green/V_{2-}, 120/Green/V_{3-}, 105/Green/V_{4-}}
      {
      
%         \draw[fill=cl\col] (\phi:2.8) circle (0.16);
%         \draw[fill=c\col, draw=none] (\phi:2.8) circle (0.04);
        \draw[fill=white, draw=none] (\phi-3.5:2.55) -- (\phi-3.5:2.9) -- (\phi+3.5:2.9) -- (\phi+3.5:2.55) arc (\phi+3.5:\phi-3.5:2.55);
        \draw(\phi-3.5:2.6) -- (\phi-3.5:2.9) -- (\phi+3.5:2.9) -- (\phi+3.5:2.6);
        \draw[fill=cl\col](\phi-2.5:2.6) -- (\phi-2.5:2.85) -- (\phi+2.5:2.85) -- (\phi+2.5:2.6) -- (\phi-2.5:2.6);
        \draw[fill=c\col, draw=c\col](\phi:2.72) circle (0.05);
        \draw(\phi:3.4) node {\color{c\col}$\x$};
      }
      
      \draw[>=latex, ->, thick, cBlue] (67.5:0) -- (67.5:1.6);
      \draw[>=latex, ->, thick, cGreen] (-52.5:0) -- (-52.5:1.6);
      \draw[>=latex, ->, thick, cRed] (-172.5:0) -- (-172.5:1.6);
      \draw[>=latex,<->, cGray](67.5:1) arc (67.5:187.5:1);
      \draw[>=latex,<->, cGray](-172.5:1) arc (-172.5:-52.5:1);
      \draw[>=latex,<->, cGray](67.5:1) arc (67.5:1-52.5:1);
      \draw[cGray] (7.5:0.6) node {$\frac{2\pi}{3}$};
      \draw[cGray] (127.5:0.6) node {$\frac{2\pi}{3}$};
      \draw[cGray] (-112.5:0.6) node {$\frac{2\pi}{3}$};
      
    \end{tikzpicture}
  \end{tabular} &
  \begin{tabular}{ll}
    \cdef{$p$} & \cdef{Polpaarzahl} \\
    \cdef{$2p$} & \cdef{Polzahl} \\
    \cdef{$N_n$} & \cdef{Nutenzahl} \\
    \cdef{m} & \cdef{Strangzahl} \\
    \cdef{q} & \cdef{Nuten pro Phasenband} \\
    \multicolumn{2}{l}{$N = 2p \cdot q \cdot m$} \\
    \multicolumn{2}{l}{$\fmm n = \frac{60 \cdot f}{p}$}
  \end{tabular}
\end{tabular}

\section{Synchronmaschine}
\begin{tabular}{ll}

  \begin{tabular}{l}
    \begin{tikzpicture}[scale=0.8, every node/.style={scale=0.8}]
      \draw[fill=clGray] (0,0) circle (4);
      \draw[fill=white] (0,0) circle (2.6);
      \draw[fill=clGray] (0,0) circle (2.5);
      
      \foreach \phi/\col in {90/Blue, 75/Blue, 60/Blue, 45/Blue,
                                -30/Green, -45/Green, -60/Green, -75/Green,
                                -150/Red, -165/Red, 180/Red, 165/Red}
      {
        \draw[fill=white, draw=none] (\phi-3.5:2.55) -- (\phi-3.5:2.9) -- (\phi+3.5:2.9) -- (\phi+3.5:2.55) arc (\phi+3.5:\phi-3.5:2.55);
        \draw(\phi-3.5:2.6) -- (\phi-3.5:2.9) -- (\phi+3.5:2.9) -- (\phi+3.5:2.6);
        \draw[fill=cl\col](\phi-2.5:2.6) -- (\phi-2.5:2.85) -- (\phi+2.5:2.85) -- (\phi+2.5:2.6) -- (\phi-2.5:2.6);
        %\draw[thick, c\col](\phi-2.5:2.6) -- (\phi+2.5:2.85);
        %\draw[thick, c\col](\phi-2.5:2.85) -- (\phi+2.5:2.6);
        \draw (\phi:2.725) node {\color{c\col}\rotatebox[origin=c]{\phi}{$\times$}};
      }
%       
      \foreach \phi/\col in {30/Red, 15/Red, 0/Red, -15/Red,
                             -90/Blue, -105/Blue, -120/Blue, -135/Blue,
                             150/Green, 135/Green, 120/Green, 105/Green}
      {
        \draw[fill=white, draw=none] (\phi-3.5:2.55) -- (\phi-3.5:2.9) -- (\phi+3.5:2.9) -- (\phi+3.5:2.55) arc (\phi+3.5:\phi-3.5:2.55);
        \draw(\phi-3.5:2.6) -- (\phi-3.5:2.9) -- (\phi+3.5:2.9) -- (\phi+3.5:2.6);
        \draw[fill=cl\col](\phi-2.5:2.6) -- (\phi-2.5:2.85) -- (\phi+2.5:2.85) -- (\phi+2.5:2.6) -- (\phi-2.5:2.6);
        \draw[fill=c\col, draw=c\col](\phi:2.72) circle (0.05);
      }

      \foreach \phi in {140, 150, 160, 170, 180, -170, -160, -150, -140}
      {
        \draw [fill=clOrange] (\phi:2.2) circle (0.115);
        \draw [fill=cOrange, draw=cOrange] (\phi:2.2) circle (0.03);
      }

      \foreach \phi in {-40, -30, -20, -10, 0, 10, 20, 30, 40}
      {
        \draw [fill=clOrange] (\phi:2.2) circle (0.115) node{$\color{cOrange}\times$};
      }
      
      
    \end{tikzpicture}
  \end{tabular} &
  \begin{tabular}{l}
    \begin{circuitikz}[scale=0.8, transform shape]
      \draw (0,2) to [short, *-, i = {\large $\cn{I_1}$}] (1,2) to [inductor, l = {\large $jX_d$}] (3,2) to [short, -*] (4,2);
      \draw (0,0) to [short, *-*] (4,0);
      \draw[-latex,shorten >=2mm, shorten <=2mm,in=110, out=250] (0,2) to node[right] {\large $\cn{U_1}$} (0,0);
      \draw[-latex,shorten >=2mm, shorten <=2mm,in=70, out=290] (4,2) to node[left] {\large $\cn{U_p}$} (4,0);
      \draw[-latex,shorten >=2mm, shorten <=2mm,in=-140, out=-40] (1,2) to node[below] {\large $\cn{U_d}$} (3,2);
    \end{circuitikz} \\
    \cdef{$X_d$: Synchronreaktanz} \\
    \cdef{$X_{\sigma 1}$: Streureaktanz} \\
    \cdef{$X_{h}$: Hauptreaktanz} \\
    \cdef{$U_p$: Polradspannung} \\
    \cdef{$U_1$: Netzspannung}
  \end{tabular}

\end{tabular}

\textbf{Meist in Stern-Schaltung!} Die Polradspannung $U_p$ ist eine fiktive Hilfsgrösse. In der
Ankerwicklung (Erreger) wird ein Gleichstrom $I_e$ angelegt, welcher das Feld erzeugt. Im Leerlauf der Maschine entspricht 
$U_p$ der von dem Erregerstrom induzierten Spannung der Statorwicklung. 
$$\cn{U_p} = \cn{U_p}(\cn{I_e}) \quad \cn{U_p} = j X_h \cdot \cn{I'_e}, \quad \text{wobei } \cn{I'_e} \text{: Erregerstrom auf Statorseite}$$

So entsteht die Grundgleichung einer Synchronmaschine:

$$\cn{U_1} = \cn{U_d} + \cn{U_p} \qquad \cn{U_d} = j X_d \cdot \cn{I_1} \qquad X_d = X_{\sigma 1} + X_h = \frac{U_1}{\sqrt{3} \cdot I_k}$$
$$\cn{U_1} = \cn{U_d} + \cn{U_p} = jX_d \cdot \cn{I_1} + \cn{U_p}(I_e)$$

\begin{tabular}{ll}
  \begin{tabular}{l}
    \begin{tikzpicture}
      \begin{axis} [
        width=0.4\columnwidth,
        height=0.35\columnwidth,
        title={Leerlauf-Kennlinie},
        axis lines=middle, 
        xlabel={$I_e$},
        ylabel={$U_1$},
        xmin = 0,
        xmax = 1,
        ymin = 0,
        ymax = 1,
        xtick = {0.15},
        ytick = {0.6},
        xticklabels = {$I_{eN}$},
        yticklabels = {$U_{1N}$},
        disabledatascaling
      ]
        \draw [draw=cRed, thick] (axis cs:0,0) -- (axis cs:0.1,0.5);
        \draw [draw=cRed, thick] (axis cs:0.1,0.5) arc [radius=0.2,start angle=0168.69,end angle=101.31]; 
        \draw [draw=cRed, thick] (axis cs:0.2569, 0.6569) -- (axis cs:1.2569, 0.8569);
        
        \draw [dashed] (axis cs:0,0.6) -- (axis cs:0.15,0.6);
        \draw [dashed] (axis cs:0.15,0) -- (axis cs:0.15,0.6);
        \draw [fill=black] (axis cs:0.15,0.6) circle (0.02);
        
      \end{axis}
    \end{tikzpicture}
  \end{tabular} &
  
  \begin{tabular}{l}
    
    \begin{tikzpicture}
      \begin{axis} [
        width=0.4\columnwidth,
        height=0.35\columnwidth,
        title={Kurzschluss-Kennlinie},
        axis lines=middle, 
        xlabel={$I_e$},
        ylabel={$I_1$},
        xmin = 0,
        xmax = 1,
        ymin = 0,
        ymax = 1,
        xtick = {0.6},
        ytick = {0.6},
        xticklabels = {$I_{eN}$},
        yticklabels = {$I_{K}$},
      ]
        \draw [draw=cBlue, thick] (axis cs:0,0) -- (axis cs:0.4,0.4);
        \draw [dashed, draw=cBlue, thick] (axis cs:0.4,0.4) -- (axis cs:0.9,0.9);
        
        \draw [dashed] (axis cs:0,0.6) -- (axis cs:0.6,0.6);
        \draw [dashed] (axis cs:0.6,0) -- (axis cs:0.6,0.6);
        \draw [fill=black] (axis cs:0.6,0.6) circle (0.02);
        
      \end{axis}
    \end{tikzpicture}
  \end{tabular}
\end{tabular}

\subsection{Zeigerdiagramme}
Bei den Zeigerdiagrammen einer Synchronmaschine wird als Referenz der Vektor 
{\color{cBlue}$\cn{U_1}$} gewählt. Danach wird {\color{cRed}$\cn{I_1}$} gesetzt.
$\cn{I'_e}$ entsteht, indem $I_e$ mit der Richtung (Umdrehung) der Welle multipliziert 
wird.

\subsubsection{Zeigerdiagramm im Motorbetrieb: $P > 0$, $\delta > 0$}

\begin{tabular}{c|c}
    Untererregt: $U_p < U_1$ (induktiv $\varphi > 0$) &
    Übererregt: $U_p > U_1$ (kapazitiv: $\varphi < 0$) \\
  \begin{tabular}{c}
    \begin{tikzpicture}
      %zeiger
      \draw [>=latex, ->, draw=cBlue, thick, line width=0.3mm] (0,0) -- (0,3) node[left]{\color{cBlue}$\cn{U_1}$}; 
      \draw [dashed] (0,0) -- (60:3.6);
      \draw [>=latex, ->, draw=cRed, thick, line width=0.3mm] (0,0) -- (60:2) node[right]{\color{cRed}$\cn{I_1}$};
      \draw [dashed] (0,3) -- + (-30:1.5);
      \draw [>=latex, <-, thick, line width=0.3mm, draw=black] (0,3) -- + (-30:1);
      \draw (0,3) node [above right] {$jX_d \cdot \cn{I_1}$};
      \draw [>=latex, ->, thick, line width=0.3mm, draw=black] (0,0) -- (0.866,2.5) node[below left, xshift=-1mm, yshift=3mm]{$\cn{U_p}$}; %70.89 Deg
      \draw [>=latex, ->, thick, line width=0.3mm, draw=black] (0,0) -- (-19.11:1.6) node[above]{$\cn{I'_e}$};
      
      %winkel
      \draw [>=latex, ->, thick, line width=0.3mm, draw=black] (60:1.5) arc (60:90:1.5) node[below right]{$\varphi$};
      \draw [>=latex, ->, thick, line width=0.3mm, draw=black] (70.89:2) arc (70.89:90:2) node[below right] {$\delta$};
      \draw [line width=0.2mm, draw=black] (60:2.3) arc (-120:-210:0.298) node [below right, xshift=-0.8mm]{$\cdot$};
      \draw [line width=0.2mm, draw=black] (-19.11:0.3) arc (-19.11:70.89:0.3) node[below, xshift=0.5mm]{$\cdot$}; 
    \end{tikzpicture} 
  
  \end{tabular} &
  \begin{tabular}{c}
    \begin{tikzpicture}
      %Zeiger
      \draw [>=latex, ->, draw=cBlue, thick, line width=0.3mm] (0,0) -- (0,3) node[below right]{\color{cBlue}$\cn{U_1}$}; 
      \draw [dashed] (0,0) -- (105:3.2);
      \draw [>=latex, ->, draw=cRed, thick, line width=0.3mm] (0,0) -- (105:2) node[left]{\color{cRed}$\cn{I_1}$};
      \draw [dashed] (0,3) -- + (-165:0.7);
      \draw [>=latex, <-, thick, line width=0.3mm, draw=black] (0,3) -- + (15:1.5) node[left, yshift=3mm]{$jX_d \cdot \cn{I_1}$};
      \draw [>=latex, ->, thick, line width=0.3mm, draw=black] (0,0) -- (1.4489, 3.3882) node[below right]{$\cn{U_p}$}; %66.85 Deg
      \draw [>=latex, ->, thick, line width=0.3mm, draw=black] (0,0) -- (-23.1527:1.6) node[above]{$\cn{I'_e}$};
      
      %Winkel
      \draw [>=latex, ->, thick, line width=0.3mm, draw=black] (105:1.5) arc (105:90:1.5) node[below left, xshift=1mm]{$\varphi$};
      \draw [>=latex, ->, thick, line width=0.3mm, draw=black] (66.85:2) arc (66.85:90:2) node[below right]{$\delta$};
      \draw [line width=0.2mm, draw=black] (105:2.598) arc (-75:15:0.3) node[below left, xshift=0.5mm]{$\cdot$};
      \draw [line width=0.2mm, draw=black] (-23.1527:0.3) arc (-23.15:66.85:0.3) node[below, xshift=0.5mm]{$\cdot$};
    \end{tikzpicture}
  \end{tabular}
\end{tabular}

\subsubsection{Zeigerdiagramm im Generatorbetrieb: $\delta < 0$}

\begin{tabular}{c|c}
  Übererregt: $U_p > U_1$ (kapatiziv: $\varphi < 0$) &
  Untererregt: $U_P < U_1$ (induktiv: $\varphi > 0$) \\
  \begin{tabular}{c}
    \begin{tikzpicture}
      \draw [>=latex, ->, draw=cBlue, thick, line width=0.3mm] (0,0) -- (90:2.5) node[right]{\color{cBlue}$\cn{U_1}$}; 
      \draw [>=latex, ->, draw=cRed,  thick, line width=0.3mm] (0,0) -- (-130:2) node[right]{\color{cRed}$\cn{I_1}$};
      \draw [dashed] (0,0) -- (50:3.2);
      \draw [>=latex, <-, draw=black, thick, line width=0.3mm] (90:2.5) -- + (140:1.2) node[right, xshift=1mm]{$jX_d \cdot \cn{I_1}$};
      \draw [dashed] (90:2.5) -- + (-40:1.6);
      \draw [>=latex, ->, draw=black, thick, line width=0.3mm] (0,0) -- (-0.9193, 3.2714) node[below left] {$\cn{U_p}$}; %105.695 deg
      \draw [>=latex, ->, draw=black, thick, line width=0.3mm] (0,0) -- (15.695:1.7) node[below]{$\cn{I'_e}$};
      
      \draw [>=latex, ->, draw=black, thick, line width=0.3mm] (-130:1.5) arc (-130:-270:1.5);
      \draw (160:1.2) node {$\varphi$};
      \draw [>=latex, ->, draw=black, thick, line width=0.3mm] (105.7:2) arc (105.7:90:2) node[below left]{$\delta$};
      \draw [line width=0.2mm, draw=black] (50:1.615) arc (-130:-220:0.3) node[below, xshift=0.5mm]{$\cdot$};
      \draw [line width=0.2mm, draw=black] (15.695:0.3) arc (15.695:105.695:0.3) node[below right, yshift = 0.4mm, xshift=-0.1mm]{$\cdot$};      
    \end{tikzpicture}
  \end{tabular} &
  \begin{tabular}{c}
    \begin{tikzpicture}
      \draw [>=latex, ->, draw=cBlue, thick, line width=0.3mm] (0,0) -- (90:3) node[right]{\color{cBlue}$\cn{U_1}$}; 
      \draw [>=latex, ->, draw=cRed,  thick, line width=0.3mm] (0,0) -- (-70:2) node[right]{\color{cRed}$\cn{I_1}$};
      \draw [dashed] (0,0) -- (110:3.3);
      \draw [>=latex, <-, draw=black, thick, line width=0.3mm] (90:3) -- + (-160:1.2) node[above, xshift=-2mm, yshift=0.5mm]{$jX_d \cdot \cn{I_1}$}; %113.51 Deg
      \draw [>=latex, ->, draw=black, thick, line width=0.3mm] (0,0) -- (-1.128,2.59) node[below left]{$\cn{U_p}$};
      \draw [>=latex, ->, draw=black, thick, line width=0.3mm] (0,0) -- (23.53:1.6) node[above]{$\cn{I'_e}$};
      
      \draw [>=latex, ->, draw=black, thick, line width=0.3mm] (-70:1) arc (-70:90:1);
      \draw (-30:0.7) node {$\varphi$};
      \draw [>=latex, ->, draw=black, thick, line width=0.3mm] (113.51:1.5) arc (113.51:90:1.5) node[below left]{$\delta$};
      \draw [line width=0.2mm, draw=black] (110:2.5) arc (-70:20:0.3) node[below left, xshift=0.5mm]{$\cdot$};
      \draw [line width=0.2mm, draw=black] (23.53:0.3) arc (23.53:113.53:0.3) node[below right, yshift = 0.4mm, xshift=-0.1mm]{$\cdot$};      
   
      
    \end{tikzpicture}
  \end{tabular}
\end{tabular}

$$(X_d \cdot I_1)^2 = \left(\frac{U_1}{\sqrt{3}}\right)^2 + \left(\frac{U_p}{\sqrt{3}}\right)^2 - 2 \cdot
\frac{U_1}{\sqrt{3}} \cdot \frac{U_p}{\sqrt{3}} \cdot \cos{\delta} $$

\subsection{Leistung}
Der Faktor $3$ kommt nur dazu, wenn die Leistung von allen 3 Strängen gefragt ist. (Leistung ist nicht abhängig von Stern- oder Dreieckschaltung.)
$$P = 3 \cdot U_1 \cdot I_1 \cdot \cos{\varphi} = 3 \cdot \frac{U_p \cdot U_1}{\omega L_d} \cdot \sin{\delta}
\qquad \cos{\varphi} = \frac{P}{3 \cdot U_1 \cdot I_1}$$ 
$$P(\delta) = 3 \cdot \frac{U_p U_1}{X_d} \cdot \sin{\delta} = \left\{ \begin{array}{ll}
P_{mech} - P_V = \omega \cdot M - P_V & \text{wenn Generator} \\
P_{mech} + P_V = \omega \cdot M + P_V & \text{wenn Motor} \\
\end{array} \right.$$

\begin{center}
  \begin{tikzpicture}
    \begin{axis} [
        width=0.7\columnwidth,
        height=0.4\columnwidth,
        axis lines=middle, 
        xlabel={$\delta$},
        ylabel={$P(\delta)$},
        xtick = {-45, 45,},
        xticklabels = {$-\frac{\pi}{4}$, $\frac{\pi}{4}$},
        ytick = {0},
        yticklabels = {},
        xmin = -150,
        xmax = 150,
        ymin = -1.4,
        ymax = 1.4,
    ]
      \addplot[draw=cRed, thick, domain=-90:90]{sin(x)};
      \draw [draw=cBlue, fill=clBlue, thick] (axis cs:-90,-1.2) rectangle (axis cs:-94,1.2);
      %\draw [draw=none, fill=white] (axis cs:-94.5,1.2) rectangle (axis cs:-150,-1.2);
      \draw [draw=cBlue, fill=clBlue, thick] (axis cs: 90,-1.2) rectangle (axis cs: 94,1.2);
      %\draw [draw=none, fill=white] (axis cs:94.5,1.2) rectangle (axis cs:150,-1.2);
      \draw (axis cs:-140,0) node [rotate=90, anchor=north]{Stabilitätsgrenze};
      \draw (axis cs:-120,0) node [rotate=90, anchor=north]{$\delta < -90$};
      \draw (axis cs: 95,0) node [rotate=90, anchor=north]{Stabilitätsgrenze};
      \draw (axis cs: 115,0) node [rotate=90, anchor=north]{$\delta > +90$};
      
    \end{axis}
  \end{tikzpicture}
\end{center}

\section{Asynchronmaschine}
\begin{tabular}{ll}

  \begin{tabular}{l}
    \begin{tikzpicture}[scale=0.8, every node/.style={scale=0.8}]
      \draw[fill=clGray] (0,0) circle (4);
      \draw[fill=white] (0,0) circle (2.6);
      \draw[fill=clGray] (0,0) circle (2.5);
      
      \foreach \phi/\col in {90/Blue, 75/Blue, 60/Blue, 45/Blue,
                                -30/Green, -45/Green, -60/Green, -75/Green,
                                -150/Red, -165/Red, 180/Red, 165/Red}
      {
        \draw[fill=white, draw=none] (\phi-3.5:2.55) -- (\phi-3.5:2.9) -- (\phi+3.5:2.9) -- (\phi+3.5:2.55) arc (\phi+3.5:\phi-3.5:2.55);
        \draw(\phi-3.5:2.6) -- (\phi-3.5:2.9) -- (\phi+3.5:2.9) -- (\phi+3.5:2.6);
        \draw[fill=cl\col](\phi-2.5:2.6) -- (\phi-2.5:2.85) -- (\phi+2.5:2.85) -- (\phi+2.5:2.6) -- (\phi-2.5:2.6);
        %\draw[thick, c\col](\phi-2.5:2.6) -- (\phi+2.5:2.85);
        %\draw[thick, c\col](\phi-2.5:2.85) -- (\phi+2.5:2.6);
        \draw (\phi:2.725) node {\color{c\col}\rotatebox[origin=c]{\phi}{$\times$}};
      }
%       
      \foreach \phi/\col in {30/Red, 15/Red, 0/Red, -15/Red,
                             -90/Blue, -105/Blue, -120/Blue, -135/Blue,
                             150/Green, 135/Green, 120/Green, 105/Green}
      {
        \draw[fill=white, draw=none] (\phi-3.5:2.55) -- (\phi-3.5:2.9) -- (\phi+3.5:2.9) -- (\phi+3.5:2.55) arc (\phi+3.5:\phi-3.5:2.55);
        \draw(\phi-3.5:2.6) -- (\phi-3.5:2.9) -- (\phi+3.5:2.9) -- (\phi+3.5:2.6);
        \draw[fill=cl\col](\phi-2.5:2.6) -- (\phi-2.5:2.85) -- (\phi+2.5:2.85) -- (\phi+2.5:2.6) -- (\phi-2.5:2.6);
        \draw[fill=c\col, draw=c\col](\phi:2.72) circle (0.05);
      }

      \foreach \phi in {5,25,...,345}
      {
        \draw[fill=white, draw=none] (\phi-2:2.55) -- (\phi-2:2.06) -- (\phi+2:2.06) -- (\phi+2:2.55);  
        \draw (\phi-2:2.5) -- (\phi-2:2.06) -- (\phi+2:2.06) -- (\phi+2:2.5);
        \draw[fill=clOrange] (\phi-1:2.4) -- (\phi+1:2.4) -- (\phi+1:2.1) -- (\phi-1:2.1) -- (\phi-1:2.4);
      }
      
      \draw (65:1.6) circle (0.12);
      \draw (65:1.6) node{$\times$}; 
      \draw (65:1.6) node[right] {$I_2$};
      \draw [>=latex, ->, draw=cBlue, shorten <=0.11cm] (65:1.6) -- ++(245:0.8) node[right] {\color{cBlue} $B_1$};
      \draw [>=latex, ->, draw=cRed,  shorten <=0.11cm] (65:1.6) -- ++(155:1)   node[below] {\color{cRed}  $F_1$};
      
      \draw [color=cGreen] (0,0) circle (0.12);
      \draw (0,0) node[right] {\color{cGreen} $M$};
      \draw (0,0) node[yshift=-0.1] {\color{cGreen} $\cdot$}; 
      
      \draw [>=latex, ->] (200:3.4) arc (200:240:3.4) node[midway, left] {$n_1$};
      \draw [>=latex, ->] (200:1.4) arc (200:240:1.4) node[midway, left] {$n$};
      
    \end{tikzpicture}
  \end{tabular} &
  \begin{tabular}{ll}
    \cdef{$n_1$}  & \cdef{Synchrone Drehzahl (Drehfeld)} \\
    \cdef{$n$}    & \cdef{Drehzahl des Läufers} \\
    \cdef{$n_2$}  & \cdef{Relative Drehzanl}\\
    \cdef{$s$}    & \cdef{Schlupf} \\
    \cdef{$I_2$}  & \cdef{Induzierter Strom} \\
    \cdef{$q_1$}  & \cdef{Anz. Phasen} \\
    \multicolumn{2}{l}{$n_2 = n_1 - n$} \\
    \multicolumn{2}{l}{$\fmm s = \frac{n_2}{n_1} = \frac{n_1 - n}{n_1} = \frac{f_2}{f_1}$} \\
    \multicolumn{2}{l}{$\fmm I'_2 = \frac{U_{1}}{\sqrt{ \left( \frac{R'_2}{s} \right)^2 +
    X_{2\sigma}'^2 }}$}
  \end{tabular}
\end{tabular}

\begin{center}
  \begin{tabular}{ll}
    \textbf{Stillstand} (Anlauf) & \textbf{Synchroner Lauf} \\
    \midrule
    $s = 1 \quad f_2 = f_1$ & $s = 0 \quad f_2 = 0$ \\
    $\fmm I_2 = I_{2_{max}} = \frac{U_{i20}}{\sqrt{ R_2'^2 + X_{2\sigma}'^2 }}$ &
    $I_2 = I_{2_{min}} = 0$
  \end{tabular}
\end{center}

\cdef{$r$: Radius des Rotors}, \cdef{$B$: Flussdichte des Drehfeldes}, \cdef{$l_2$: Länge des Rotors}
$$M = r \cdot F_1 = R \cdot I_2 B l_2 = \frac{60}{2\pi n_1} \cdot \frac{P_{Cu2}}{s} \qquad \frac{M}{M_K} =
\frac{2}{\frac{s}{s_k} + \frac{s_k}{s}} \qquad s_k = \frac{R'_2}{X'_{2\sigma}}$$ 
$$M = \frac{60}{2 \pi \cdot n_1} \frac{P_{Cu2}}{s} = 
\frac{60 \cdot q_2 \cdot U_{i20}^2 \cdot R'_2}{2 \pi n_1 s \cdot \left(\left(\frac{R'_2}{s}\right)^2 + X_{2\sigma}'^2\right)} \qquad 
M_K = \frac{60 \cdot q_1}{4 \pi n_1} \cdot \frac{U_1^2}{X'_{2\sigma}}$$


\subsection{Leistung}
%\cdef{$P_1$: primäre Netzleistung,  
%$P_{D1}$: Drehfeldleistung, $P_m$: mechanische Leistung, $P_R$: Reibungsverluste und Lüftung,
%$P_{m}'$: mechanische Nutzleistung}

$$P_m' = P_m - P_R = P_{D1} - P_{Cu2} - P_R = P_1 - P_{Cu1} - P_{Fe} - P_{Cu2} - P_R $$
$$P_{D1} = \frac{2\pi}{60} \cdot n_1 \cdot M \qquad P_{m} = \frac{2\pi}{60} \cdot n \cdot M$$

\begin{center}
  \begin{tikzpicture}
    \begin{axis} [
      xmin = -1.6,
      xmax = 1.6,
      ymin = -1.2,
      ymax = 1.2,
      xlabel={$s$},
      ylabel={$M$},
      width=0.9\columnwidth,
      height=0.6\columnwidth,
      axis lines=middle,
      x dir=reverse,
      xtick={-1,-0.5,-0.2,0.2,0.5,1},
      xticklabels={-1,-0.5,$-s_k$,$s_k$,0.5,1},
      ytick={-1},
      yticklabels={$-M_k$},
      extra y ticks = {0.3846, 1},
      extra y tick labels = {$M_A$,$M_K$},
      extra y tick style = {y tick label style={right}},
      every axis y label/.style={
        at={(ticklabel* cs:1)},
        anchor=south,
      },
    ]
      
      \draw [draw=none, fill=clRed, opacity=0.5] (axis cs:1,0) rectangle (axis cs:1.6,1.2);
      \draw (axis cs:1.3,1) node {Bremse};
      
      \draw [draw=none, fill=clBlue, opacity=0.5] (axis cs:1,0) rectangle (axis cs:0,1.2);
      \draw (axis cs:0.7,1) node {Motor};
      
      \draw [draw=none, fill=clGreen, opacity=0.5] (axis cs:0,0) rectangle (axis cs:-1.6,-1.2);
      \draw (axis cs:-1.2,-1) node{Generator};
      
      \addplot[cRed,   thick, domain=-1.6:0.02, samples=150] {2/((x/0.2)+(0.2/x))};
      \addplot[cGreen, thick, domain=0.02:0.08, samples=150] {2/((x/0.2)+(0.2/x))};
      \addplot[cRed,   thick, domain=0.08:1.6,  samples=150] {2/((x/0.2)+(0.2/x))};
      
      \draw [dashed] (axis cs:1,0) -- (axis cs:1,0.3846) -- (axis cs:0,0.3846);
      \draw [cRed, fill=cRed] (axis cs:1,0.3846) circle (1mm);
      
      \draw [dashed] (axis cs:0.2,0) -- (axis cs:0.2,1) -- (axis cs:0,1);
      \draw [cRed, fill=cRed] (axis cs:0.2,1) circle (1mm);
      
      \draw [>=latex, ->] (axis cs:0.8,0.55) arc (-70:-55:2.6cm) node[midway, above, rotate=25]{Anlauf};
      
      \draw [>=latex, ->, cGreen] (axis cs:-0.5,0.8) node[right] {\color{cGreen} Normalbetrieb} -- (axis cs:0.04,0.5);
    
    \end{axis}
  \end{tikzpicture}
  
\cdef{$M_k$: Kippmoment, $M_A$: Anlaufmoment, $s_k$: Kippschlupf}
\end{center}

\subsection{Modell der Asynchronmaschine}

\begin{center}
  \begin{circuitikz} [scale=0.8, transform shape]
    \draw (0,3) to [short, o-, i={\large $\cn{I_1}$}] (1,3) to [R, l={\large $R_1$}] (3,3) 
      to [L, l={\large $X_{\sigma 1}$}] (5,3) to [short] (8,3) 
      to [L, l={\large $X'_{\sigma 2}$}] (10,3) to [short, i={\large $\cn{I_2'}$}] (11,3) 
      to [R, l={\large $\frac{R_2'}{s}$}] (11,0) to [short, -o] (0,0);
    \draw (5.5,0) to [R, l^={\large $R_{Fe}$}, i<^={\large $\cn{I_{Fe}}$}, *-*] (5.5,3);
    \draw (6.5,0) to [L, l_={\large $X_{h}$}, i<_={\large $\cn{I_{\mu}}$}, *-*] (6.5,3);
    \draw [>=latex, ->, bend left=20, shorten >=2mm, shorten <=2mm](7.5,3) to node[right] {\large $\cn{U_h}$} (7.5,0); 
    \draw [>=latex, ->, bend right=20, shorten >=2mm, shorten <=2mm](0,3) to node[left] {\large $\cn{U_1}$} (0,0); 
  \end{circuitikz}
\end{center}

\begin{tabular}{ll}
  \begin{dtabular}
    $N$ & Windungszahl \\
    $k_w$ & Wicklungsfaktor \\
    $R_{Fe}$ & Eisen-Verlustwiederstand \\
    $X_{h}$ & Hauptreaktanz \\
    $\cn{U_h}$ & innere Spannung   \\
    $\cn{I_{\mu}}$ & Magnetisierungsstrom \\
    $u$ & Übersetzungsverhältnis \\
    $I_0$ & Leerlaufstrom \\
  \end{dtabular} &
  \begin{mtabular}{l}
    $\fmm \cn{I_1} = \cn{I_{Fe}} + \cn{I_\mu} + \cn{I_{2}'} \approx \cn{I_{2}'}$ \\
    $\fmm \cn{U_1} = R_1 \cdot \cn{I_1} + j X_{\sigma 1} \cn{I_1} + \cn{U_h}$ \\
    $\fmm u = \frac{N_1 \cdot k_{w1}}{N_2 \cdot k_{w2}}, \quad \cn{I_2'} = \cn{I_2} \cdot u, 
      \quad  R_{2}' = R_2 \cdot u^2$ \\
    $\fmm \cn{I_0} = \cn{I_{Fe}} + \cn{I_\mu} \qquad \cos{\varphi} = \frac{P}{U \cdot I}$ \\
    $\fmm s_k = \frac{R'_2 }{X'_{2\sigma}} \qquad M_K = \frac{60 \cdot q_1}{4\pi n_1} \cdot \frac{U_1^2}{X'_{2\sigma}}$ \\
    $\fmm U_{Y} = \frac{U_{\Delta}}{\sqrt{3}} \quad I_{Y} = \frac{I_{\Delta}}{\sqrt{3}} \quad M_{Y} = \frac{M_{\Delta}}{3}$    
  \end{mtabular}
\end{tabular}

\subsection{Zeigerdiagramm}
\begin{center}
  \begin{tikzpicture}
    \draw [>=latex, ->, thick, line width=0.3mm] (0,0) -- (0,3) node[above]{$\cn{U_1}$}; 
    \draw [>=latex, ->, thick, line width=0.3mm, color=cRed]  (0,0) -- (10:1) node[above] {\color{cRed} $\cn{I_0}$};
    \draw [>=latex, ->, thick, line width=0.3mm, color=cBlue] (0,0) -- (-15:0.906) node[below left] {\color{cBlue} $\cn{I_{\mu}}$};
    \draw [>=latex, ->, thick, line width=0.3mm, color=cBlue] (-15:0.906) -- (10:1) node[below right] {\color{cBlue} $\cn{I_{Fe}}$};
    \draw [>=latex, ->, thick, line width=0.3mm, color=cGreen] (0,0) -- (40:2.3) node[above] {\color{cGreen} $\cn{I_1}$};
    \draw [>=latex, ->, thick, line width=0.3mm, color=cGreen] (10:1) -- (40:2.3) node[below right] {\color{cGreen} $\cn{I'_2}$};
    \draw [>=latex, ->, thick, line width=0.3mm] (0,0) -- (75:2) node[right]{$\cn{U_h}$}; 
    \draw [>=latex, <-, thick, line width=0.3mm] (0,3) -- ++ (220:0.3) node[above left] {$R_1 \cdot \cn{I_1}$} ;
    \draw [>=latex, ->, thick, line width=0.3mm] (75:2) -- (94.68:2.817) node[below left] {$j X_{\sigma 1} \cdot \cn{I_1}$};
 
    %angles
    \draw [line width=0.2mm] (-15:0.3) arc (-15:75:0.3) node[below,xshift=1mm] {$\cdot$};
    \draw [>=latex, ->, line width=0.2mm] (90:1.2) arc (90:40:1.2) node[midway, below] {$\varphi$};
  \end{tikzpicture}
\end{center}

\subsection{Leerlauf}
Hier wird die Asynchronmaschine an der Welle nicht belastet. 

\begin{tabular}{ll}
  \begin{mtabular}{l}
    $\fmm R_1 \ll R_{Fe}, \quad X_{\sigma 1} \ll X_h$ \\
    $\fmm R_{Fe} = \frac{U_0}{I_{Fe}} = \frac{U_0}{I_0 \cdot \cos{\varphi_0}}$ \\
    $\fmm X_{h} = \frac{U_0}{I_{\mu}} = \frac{U_0}{I_0 \cdot \sin{\varphi_0}}$ \\
  \end{mtabular} &
  \begin{tabular}{l}
    \begin{circuitikz} [scale=0.8, transform shape]
    \draw (0,2.5) to [short, o-, i={\large $\cn{I_1}$}] (3,2.5) to [short] (4,2.5);
    \draw (3,0) to [R, l^={\large $R_{Fe}$}, i<^={\large $\cn{I_{Fe}}$}, *-*] (3,2.5);
    \draw (4,0) to [L, l_={\large $X_{h}$}, i<_={\large $\cn{I_{\mu}}$}] (4,2.5);
    \draw (4,0) to [short, -o] (0,0);
    \draw [>=latex, ->, bend right=20, shorten >=2mm, shorten <=2mm](0,2.5) to node[left] {\large $\cn{U_0}$} (0,0); 
    \end{circuitikz}
  \end{tabular}
\end{tabular}

\subsection{Kurzschluss}
Hier wird die Welle der Asynchronmaschine blokiert ($s = 1$).

\begin{center}
  \begin{circuitikz} [scale=0.9, transform shape]
    \draw (0,2) to [short, o-, i={\large $\cn{I_k}$}] (1,2) to [R, l={\large $R_1$}] (3,2)
      to [L, l={\large $X_{\sigma 1}$}] (5,2) to [short] (6,2) to [L, l={\large $X_{\sigma 2}'$}] (8,2)
      to [short, i={\large $\cn{I'_2}$}] (9,2) to [R, l={\large $R'_2$}] (9,0) to [short, -o] (0,0);
    \draw [>=latex, ->, bend right=20, shorten >=2mm, shorten <=2mm] (0,2) to node[left] {\large
    $\cn{U_k}$} (0,0);
  \end{circuitikz}
\end{center}

$$R_1 + R'_2 = \frac{U_R}{I_K} = \frac{U_K \cdot \cos{\varphi_K}}{I_K} \qquad 
X_{\sigma 1} + X'_{\sigma 2} = \frac{U_X}{I_K} = \frac{U_K \cdot \sin{\varphi_k}}{I_K}$$
$$\cos{\varphi_K} = \frac{P_K}{U_K \cdot I_K} \qquad X'_{2\sigma} = \frac{q_1 \cdot 60}{4 \pi n_1} \cdot
\frac{U^2_1}{M_K}$$

\section{Schrittmotor}

\begin{tabular}{ll}

  \begin{tabular}{l}
    \begin{tikzpicture} [scale=0.8, every node/.style={scale=0.8}]
      %Kern
      \draw [fill=clGray] (0,0) circle (4);
      \draw (0,3.4) node{Ständer}; 
      \draw [fill=white] (0,0) circle (3);
      
      %Zähne
      \draw [fill=clGray, draw=none] (-0.5,3) -- (0,3.3) -- (0.5,3) -- (0.5,1.75) arc (74.055:105.95:1.82) -- (-0.5,3);
      \draw (0.5,2.958) -- (0.5,1.75) arc (74.055:105.95:1.82) -- (-0.5,2.958);
      
      \draw [fill=clGray, draw=none] (-0.5,-3) -- (0,-3.3) -- (0.5,-3) -- (0.5,-1.75) arc (-74.055:-105.95:1.82) -- (-0.5,-3);
      \draw (0.5,-2.958) -- (0.5,-1.75) arc (-74.055:-105.95:1.82) -- (-0.5,-2.958);
      
      \draw [fill=clGray, draw=none] (3, -0.5) -- (3.3,0) -- (3,0.5) -- (1.75,0.5) arc (15.945:-15.945:1.82) -- (3,-0.5);
      \draw (2.958,0.5) -- (1.75,0.5) arc (15.945:-15.945:1.82) -- (2.958,-0.5); 
      
      \draw [fill=clGray, draw=none] (-3, -0.5) -- (-3.3,0) -- (-3,0.5) -- (-1.75,0.5) arc (164.055:195.945:1.82) -- (-3,-0.5);
      \draw (-2.958,0.5) -- (-1.75,0.5) arc (164.055:195.945:1.82) -- (-2.958,-0.5);      
      
      %Wicklungen
      \draw[preaction={fill, clRed}, pattern=north west lines, pattern color=cRed] (0.55, 2.05) rectangle (1.15,2.65);
      \draw[preaction={fill, clRed}, pattern=north west lines, pattern color=cRed] (-0.55, 2.05) rectangle (-1.15,2.65);
      \draw[preaction={fill, clRed}, pattern=north west lines, pattern color=cRed] (0.55, -2.05) rectangle (1.15,-2.65);
      \draw[preaction={fill, clRed}, pattern=north west lines, pattern color=cRed] (-0.55, -2.05) rectangle (-1.15,-2.65);
      
      \draw[preaction={fill, clBlue}, pattern=north west lines, pattern color=cBlue] (2.05,0.55) rectangle (2.65,1.15);
      \draw[preaction={fill, clBlue}, pattern=north west lines, pattern color=cBlue] (2.05,-0.55) rectangle (2.65,-1.15);
      \draw[preaction={fill, clBlue}, pattern=north west lines, pattern color=cBlue] (-2.05,0.55) rectangle (-2.65,1.15);
      \draw[preaction={fill, clBlue}, pattern=north west lines, pattern color=cBlue] (-2.05,-0.55) rectangle (-2.65,-1.15);
      
      %rotor
      \draw [fill=clGray, draw=black] (-80:1.65) arc (-80:-42.364:1.65) -- (100:1.65) arc (100:137.638:1.65) -- (-80:1.65);
      \draw (0,0) node [rotate=-61.181] {Rotor};
      
      %nodes
      \draw (90:2.25) circle (0.2) node{1};
      \draw (180:2.25) circle (0.2) node{2};
      \draw (-90:2.25) circle (0.2) node{3};
      \draw (0:2.25) circle (0.2) node{4};
      
      \draw (118.819:1.2) circle (0.2) node{5};
      \draw (-61.181:1.2) circle (0.2) node{6};
      
      %achsen
      \draw [cOrange] (118.819:1.65) -- (118.819:2.2) node[left, yshift=-3mm] {d-Achse};
      \draw [cOrange] (-61.181:1.65) -- (-61.181:2.2);
        
      \draw [cOrange] (-151.181:0.53) -- (-151.181:1.8) node[below, xshift=5mm] {q-Achse};
      \draw [cOrange] (28.819:0.53) -- (28.819:1.8);
      
      %angle
      \draw [cOrange] (0:0.61) -- (0:1.5);
      \draw [cOrange] (0:1.2) arc (0:28.819:1.2) node[midway, left, yshift=-1mm] {$\gamma_r$};
      
    \end{tikzpicture}
  \end{tabular} &
  \begin{tabular}{l}
    \circled{1}: Statorzahn- und Wicklung 1 \\
    \circled{2}: Statorzahn- und Wicklung 2 \\
    \circled{3}: Statorzahn- und Wicklung 3 \\
    \circled{4}: Statorzahn- und Wicklung 4 \\
    \circled{5}: Rotorzahn 1 \\
    \circled{6}: Rotorzahn 2 \\
  \end{tabular}
\end{tabular}

Diese Grafik ist nicht realistisch. Normalerweise sind viel mehr Statorzähne vorhanden.
Der Rotor kann aus Eisen, sowie aus einem Permamentmagnet bestehen. Wenn er ein
Permamentmagnet ist, so kann dieser auch abstossend wirken. So kann die Richtung gewechselt
werden.

\begin{tabular}{ll}
  \begin{dtabular}
    $Z_s$ & Stator-Zahnzahl \\
    $Z_R$ & Rotor-Zahnzahl \\
    $\alpha_S$ & Stator-Winkel \\
    $\alpha_R$ & Rotor-Winkel \\
    $\alpha_0$ & Vollschritt-Winkel \\
    $m$ & Strangzahl \\
    $N_p$ & Schrittzahl \\
    $f_s$ & Steuerfrequenz \\
    $L_d$ & Ind. in d-Achse \\
    $L_q$ & Ind. in q-Achse \\
    $A_Z$ & Fläche eines Zahns \\
  \end{dtabular} &
  \begin{mtabular}{l}
    $\fmm \alpha_S = \frac{2 \pi}{Z_S} \qquad \alpha_R = \frac{2 \pi}{Z_R}$ \\
    $\fmm \alpha_0 = \alpha_R - \alpha_S \qquad m = \frac{Z_S}{Z_S - Z_R}$ \\
    $\fmm N_p = \frac{2\pi}{\alpha_0} \qquad f_s = N_p \cdot \frac{n}{60}$ \\
    $\fmm L_d = 2 N \cdot \frac{\Phi}{I_1} = 2N^2 \cdot \mu_0 \cdot \frac{A_Z}{\delta_d}$ \\
    $\fmm L_q = 2 N \cdot \frac{\Phi}{I_1} = 2N^2 \cdot \mu_0 \cdot \frac{A_Z}{\delta_q}$ \\
  \end{mtabular}
\end{tabular}

\cdef{$\omega_1$} ist die Geometrische Kreisfrequenz des Rotors zu dem Zeitpunkt des
Impulseinschaltens ($\omega_1 = 0$ bedeutet der Rotor ist im Stillstand, ist $\omega_1 =
\omega_{BP)$, dann ist man am Betriebspunkt).

\subsection{Drehmoment und Leistung}

\begin{tabular}{ll}
  \begin{dtabular}
    $W_m$ & mechanische Energie \\
    $L_d$ & Induktivität in d-Achse \\
    $L_q$ & Induktivität in q-Achse \\
    $\delta_d$ & Kleinster Luftspalt \\
    $\delta_q$ & grösster Luftspalt \\
    $M_M$ & Motormoment \\
    $J_g$ & gesamtes Trägheitsmoment \\
    $N$ & Windungszahl einer Spule \\
  \end{dtabular} &
  \begin{mtabular}{l}
    $\fmm M = \frac{dW_m}{d\varphi} = \frac{1}{2}\cdot I^2 \cdot \frac{dL(\varphi)}{d\varphi}$ \\
    $\fmm \Rightarrow M_M = \frac{1}{2}\cdot I^2 \cdot \frac{L_d - L_q}{\alpha_0}$ \\
    $\fmm L_d = N^2 \cdot \mu_0 \cdot \frac{A_z}{2\delta_d}$ \\
    $\fmm L_q = N^2 \cdot \mu_0 \cdot \frac{A_z}{2\delta_q}$ \\
    $\fmm M_M = J_g \cdot \frac{\omega_s - \omega_1}{f_s} + M_L = J_g \cdot \dot{\omega} + M_L$
  \end{mtabular}
\end{tabular}

\begin{center}
  \begin{tikzpicture}
    \draw [draw=none, fill=clGray] (0,2.2) -- (1,2.2) -- (1,1.2) -- (1.6,1.2) -- (1.6,2.2) -- (2.6,2.2) -- (2.6,1.2) -- (3.2,1.2) -- (3.2,2.2) -- (4.2,2.2) -- (4.2,1.2) -- (4.8,1.2) -- (4.8,2.2) -- (5.8,2.2) -- (5.8,1.2) -- (6.4,1.2) -- (6.4,2.2) -- (7.4,2.2) -- (7.4,3) -- (0,3) -- (0,2.2); 
    \draw (0,2.2) -- (1,2.2) -- (1,1.2) -- (1.6,1.2) -- (1.6,2.2) -- (2.6,2.2) -- (2.6,1.2) -- (3.2,1.2) -- (3.2,2.2) -- (4.2,2.2) -- (4.2,1.2) -- (4.8,1.2) -- (4.8,2.2) -- (5.8,2.2) -- (5.8,1.2) -- (6.4,1.2) -- (6.4,2.2) -- (7.4,2.2); 
    \draw (6,2.6) node {Stator};
    \draw[preaction={fill, clRed}, pattern=north west lines, pattern color=cRed] (0.95,2) rectangle (0.65,1.4);
    \draw[preaction={fill, clRed}, pattern=north west lines, pattern color=cRed] (1.65,2) rectangle (1.95,1.4);
    \draw[preaction={fill, clRed}, pattern=north west lines, pattern color=cRed] (4.15,2) rectangle (3.85,1.4);
    \draw[preaction={fill, clRed}, pattern=north west lines, pattern color=cRed] (4.85,2) rectangle (5.15,1.4);
      
    \draw [dashed] (1.3,1) -- (0.5,1) (0.5,1.2) -- (1,1.2);
    \draw [>=latex, ->] (0.5,0.6) -- (0.5,1);
    \draw [>=latex, ->] (0.5,1.6) -- (0.5,1.2);
    \draw (0.5,1.1) node[left] {$\delta_d$};
    
    \draw [>=latex, <->] (6.1,1.2) -- (6.1,0);
    \draw (6.1,0.6) node[right] {$\delta_q$};
    
    
    \draw [draw=none, fill=clGray] (0,0) -- (1.3,0) -- (1.3,1) -- (1.9,1) -- (1.9,0) -- (4.5,0) -- (4.5,1) -- (5.1,1) -- (5.1,0) -- (7.4,0) -- (7.4,-0.8) -- (0,-0.8) -- (0,0); 
    \draw (0,0) -- (1.3,0) -- (1.3,1) -- (1.9,1) -- (1.9,0) -- (4.5,0) -- (4.5,1) -- (5.1,1) -- (5.1,0) -- (7.4,0); 
    \draw (6,-0.4) node{Rotor};
    
    
   
  \end{tikzpicture}
\end{center}

\begin{center}
  \begin{tikzpicture}
    \begin{axis} [
      xmin = 0.996,
      xmax = 5.29,
      ymin = 0,
      ymax = 1.4,
      xlabel={$\gamma_r$},
      ylabel={$L$},
      width=0.6\columnwidth,
      height=0.3\columnwidth,
      axis lines=middle,
      xtick={3.142},
      xticklabels={$\alpha_0$},
      ytick={0.2,1.2},
      yticklabels={$L_q$, $L_d$},
      legend pos=outer north east,
      legend style={draw=none},
    ]
      
      \addplot [thick, color=cBlue] coordinates {
        (0.997,1.2)(3.142,0.2)(5.29,1.2)
      };
      \addlegendentry{Lineare Annäherung};
      
      \addplot [thick, color=cRed] coordinates {
        (0.99700,1.20000)(1.01857,1.19942)(1.04015,1.19773)(1.06172,1.19496)(1.08329,1.19113)(1.10486,1.18627)(1.12644,1.18040)(1.14801,1.17354)(1.16958,1.16574)(1.19116,1.15702)(1.21273,1.14742)(1.23430,1.13697)(1.25587,1.12571)(1.27745,1.11368)(1.29902,1.10091)(1.32059,1.08745)(1.34217,1.07332)(1.36374,1.05858)(1.38531,1.04327)(1.40688,1.02742)(1.42846,1.01107)(1.45003,0.99427)(1.47160,0.97706)(1.49318,0.95947)(1.51475,0.94156)(1.53632,0.92335)(1.55789,0.90489)(1.57947,0.88622)(1.60104,0.86737)(1.62261,0.84838)(1.64419,0.82929)(1.66576,0.81014)(1.68733,0.79096)(1.70890,0.77178)(1.73048,0.75264)(1.75205,0.73356)(1.77362,0.71459)(1.79520,0.69574)(1.81677,0.67705)(1.83834,0.65855)(1.85991,0.64025)(1.88149,0.62218)(1.90306,0.60437)(1.92463,0.58684)(1.94621,0.56961)(1.96778,0.55269)(1.98935,0.53611)(2.01092,0.51988)(2.03250,0.50402)(2.05407,0.48853)(2.07564,0.47344)(2.09722,0.45875)(2.11879,0.44448)(2.14036,0.43063)(2.16193,0.41720)(2.18351,0.40421)(2.20508,0.39167)(2.22665,0.37956)(2.24823,0.36791)(2.26980,0.35670)(2.29137,0.34594)(2.31294,0.33563)(2.33452,0.32577)(2.35609,0.31635)(2.37766,0.30738)(2.39924,0.29884)(2.42081,0.29073)(2.44238,0.28305)(2.46395,0.27580)(2.48553,0.26895)(2.50710,0.26251)(2.52867,0.25647)(2.55025,0.25081)(2.57182,0.24553)(2.59339,0.24062)(2.61496,0.23607)(2.63654,0.23186)(2.65811,0.22799)(2.67968,0.22444)(2.70126,0.22119)(2.72283,0.21825)(2.74440,0.21558)(2.76597,0.21319)(2.78755,0.21105)(2.80912,0.20916)(2.83069,0.20749)(2.85227,0.20604)(2.87384,0.20479)(2.89541,0.20373)(2.91698,0.20283)(2.93856,0.20209)(2.96013,0.20149)(2.98170,0.20102)(3.00328,0.20066)(3.02485,0.20040)(3.04642,0.20022)(3.06799,0.20010)(3.08957,0.20004)(3.11114,0.20001)(3.13271,0.20000)(3.15429,0.20000)(3.17586,0.20001)(3.19743,0.20004)(3.21901,0.20012)(3.24058,0.20024)(3.26215,0.20044)(3.28372,0.20072)(3.30530,0.20110)(3.32687,0.20159)(3.34844,0.20221)(3.37002,0.20298)(3.39159,0.20390)(3.41316,0.20500)(3.43473,0.20628)(3.45631,0.20777)(3.47788,0.20948)(3.49945,0.21141)(3.52103,0.21359)(3.54260,0.21603)(3.56417,0.21875)(3.58574,0.22174)(3.60732,0.22504)(3.62889,0.22865)(3.65046,0.23258)(3.67204,0.23685)(3.69361,0.24147)(3.71518,0.24644)(3.73675,0.25178)(3.75833,0.25751)(3.77990,0.26362)(3.80147,0.27013)(3.82305,0.27705)(3.84462,0.28438)(3.86619,0.29213)(3.88776,0.30032)(3.90934,0.30893)(3.93091,0.31798)(3.95248,0.32748)(3.97406,0.33742)(3.99563,0.34781)(4.01720,0.35865)(4.03877,0.36994)(4.06035,0.38167)(4.08192,0.39385)(4.10349,0.40648)(4.12507,0.41954)(4.14664,0.43304)(4.16821,0.44697)(4.18978,0.46132)(4.21136,0.47608)(4.23293,0.49124)(4.25450,0.50679)(4.27608,0.52272)(4.29765,0.53902)(4.31922,0.55566)(4.34079,0.57263)(4.36237,0.58992)(4.38394,0.60750)(4.40551,0.62536)(4.42709,0.64347)(4.44866,0.66180)(4.47023,0.68034)(4.49180,0.69907)(4.51338,0.71794)(4.53495,0.73693)(4.55652,0.75602)(4.57810,0.77517)(4.59967,0.79435)(4.62124,0.81353)(4.64281,0.83267)(4.66439,0.85175)(4.68596,0.87071)(4.70753,0.88953)(4.72911,0.90817)(4.75068,0.92659)(4.77225,0.94475)(4.79382,0.96261)(4.81540,0.98013)(4.83697,0.99727)(4.85854,1.01399)(4.88012,1.03026)(4.90169,1.04602)(4.92326,1.06123)(4.94483,1.07587)(4.96641,1.08988)(4.98798,1.10322)(5.00955,1.11586)(5.03113,1.12776)(5.05270,1.13888)(5.07427,1.14918)(5.09584,1.15863)(5.11742,1.16719)(5.13899,1.17482)(5.16056,1.18151)(5.18214,1.18720)(5.20371,1.19189)(5.22528,1.19553)(5.24685,1.19811)(5.26843,1.19960)(5.29000,1.19999)
      };
      \addlegendentry{Wahre Induktivität};
      
      
      \draw [dashed] (axis cs:0,0.2) -- (axis cs:6.2,0.2);
      \draw [dashed] (axis cs:0,1.2) -- (axis cs:6.2,1.2);
      \draw [dashed] (axis cs:3.142,0) -- (axis cs:3.142,1.4);
            
    \end{axis}
  \end{tikzpicture}
\end{center}

  $$p_\delta(t) = \frac{1}{2} \cdot \frac{dL}{d\gamma_r}(\gamma_r) \cdot \omega_r \cdot i^2(t)$$
  $$M_M(t) = \frac{p_\delta}{\omega_r}(t) = \frac{1}{2} \cdot \frac{dL}{d\gamma_r}(\gamma_r)
  \cdot i^2(t)$$ $$M_L (f_s, \omega_1) = M_M - J_g \alpha_0 f_s^2 + J_g \omega_1 f_s$$
  $$P_{Stator}(t) = R i^2(t) + \frac{\partial L}{\partial \gamma_r}(\gamma_r) \cdot \omega_r \cdot i^2(t) +
  L i(t) \cdot \frac{di}{dt}(t)$$

\section{Vergleich der Maschinen}
\begin{tabular}{|c|c|c|c|c|}
 \hline
 %& GSM & RSM & SYM & ASM \\
 & GSM & RSM & SYM & ASM \\ \hline
 Komplexität des Aufbaus & \clCell{Red}{4} & \clCell{Blue}{2} & \clCell{Orange}{3} & \clCell{Green}{1} \\ \hline
 Kosten & \clCell{Red}{4} & \clCell{Orange}{3} & \clCell{Blue}{2} & \clCell{Green}{1} \\ \hline
 Wirkungsgrad & \clCell{Red}{4} & \clCell{Blue}{2} & \clCell{Green}{1} & \clCell{Orange}{3} \\ \hline
 Anpassungsfähigkeit & \clCell{Green}{1} & \clCell{Blue}{2} & \clCell{Orange}{3} & \clCell{Red}{4} \\ \hline 
\end{tabular}

\subsection{Anlaufstrom}
\begin{itemize}
  \item \textbf{GSM}: Begrenzung durch Vorwiederstand: $I_a = \frac{U}{R_a}$
  \item \textbf{RSM}: Begrenzung durch Speisung
  \item \textbf{SYM}: $I_a = K \cdot I_n, \quad 0.5 \leq K \leq 30$
  \item \textbf{ASM}: $I_a = K \cdot I_n, \quad 0.5 \leq K \leq 30$
\end{itemize}

\subsection{Anlaufmoment}
\begin{itemize}
  \item \textbf{GSM}: $M \sim I$, Begrenzt durch $R_a$, $I_{max}$
  \item \textbf{RSM}: $M \sim U$
  \item \textbf{SYM}: $\fmm M \sim \left( \frac{U}{f} \right)^2$, Begrenzt durch $X_\sigma$
  \item \textbf{ASM}: $\fmm M \sim \left( \frac{U}{f} \right)^2$, Begrenzt durch $X_\sigma$
\end{itemize}

\subsection{Drehzahlregelung}
\begin{itemize}
  \item \textbf{GSM}: Über Spannung und Erregerstrom
  \item \textbf{RSM}: Über die Frequenz der digitalen Logik
  \item \textbf{SYM}: Über die Poolparzahl und die Frequenz
  \item \textbf{ASM}: Über die Poolparzahl und die Frequenz
\end{itemize}

\subsection{Anwendungsbereiche}
\begin{itemize}
  \item \textbf{GSM}: Regelbare Antriebe mit grossem Stellbereich und guter Dynamik
  \item \textbf{RSM}: Verstellantriebe kleiner Leistung ohne Regelung
  \item \textbf{SYM}: Antriebe mit konstanter Drehzahl und gutem Leistungsfaktor
  \item \textbf{ASM}: Einfache Antriebe und regelbare Antriebe mit beschränkter Dynamik
\end{itemize}

\end{twocolumn}

\end{document}
